% Created 2016-04-21 Thu 10:09
\documentclass[11pt]{article}
\usepackage[utf8]{inputenc}
\usepackage[T1]{fontenc}
\usepackage{fixltx2e}
\usepackage{graphicx}
\usepackage{grffile}
\usepackage{longtable}
\usepackage{wrapfig}
\usepackage{rotating}
\usepackage[normalem]{ulem}
\usepackage{amsmath}
\usepackage{textcomp}
\usepackage{amssymb}
\usepackage{capt-of}
\usepackage{hyperref}
\author{liweilijie}
\date{}
\title{以太坊智能彩票系统问题汇总}
\hypersetup{
 pdfauthor={liweilijie},
 pdftitle={以太坊智能彩票系统问题汇总},
 pdfkeywords={},
 pdfsubject={},
 pdfcreator={Emacs 24.5.1 (Org mode 8.3.4)}, 
 pdflang={English}}
\begin{document}

\maketitle
\tableofcontents


\section{一些待定的问题}
\label{sec:orgheadline1}
\begin{itemize}
\item 智能合约参数设定,一些有关游戏规则的动态变化如何来设定。
\item 智能合约执行的时候检测,需要保存用户信息。如何来保存。
\item 如何做到智能开奖,智能合约能否做到?假如有1万个用户下注,智能合约脚本可以获取到用户信息吗?
\item 如何兑奖,涉及到多个帐户转帐。比如1万个帐户。
\item 钱包的问题,一个用户在中心网站注册生成一个下注地址,涉及到如何通过程序创建钱包。以及钱包的管理问题。
\item 闪电网络现在还没有成熟起来,后续我们如何用闪电网络的问题?
\item 每一个用户下注的时候在智能合约里面就知道用户的Address地址,下了多少注,下注内容,这些信息都需要保存起来。如果用户量比较大的时候。这时候会不会有问题?
\end{itemize}


\section{上海交流会以太坊问题}
\label{sec:orgheadline2}

\#+CAPTION:以太坊问题
\begin{center}
\begin{tabular}{ll}
知识点 & 具体内容\\
\hline
storage & 智能合约里面保存数据的容量有没有限制\\
 & storage的使用是如何收费的\\
 & 查询storage里面的信息是如何收费的,能不能有办法归避收费的问题\\
 & 如果信息量过大的话,查询效率能不能保证\\
 & 有没有接口能将storage里面的数据导出来。导出来会不会收费\\
\hline
合约相关 & 智能合约有没有一个timer定时器的功能,做到定时触发一个机制去做某件事情\\
 & 合约到期到,如果不需要此合约了,如何清除此合约。所谓的清除是其他节点再来调用则不会成功\\
 & 合约清除的时候 \textbf{msg.sender.send(this.balance)} 和 \textbf{suicide(owner)} 有什么不一样,是都要调用吗?\\
 & 作为一个普通用户,如果查看到智能合约相关的源代码。并且通过查看源代码的内容相信此合约没有作假行为\\
 & 作为一个节点用户,除了通过合约地址与合约的ABI调用某个合约之外,还有没有其他方式调用合约的接口\\
 & 当调用合约失败了,如何才能给用户一个比较友好的提示,让用户知道调用合约失败。失败原因提示给用户\\
 & 调用合约除了geth钱包外,rpc外,还有没有其他办法,另外rpc调用是如何调用的。钱包geth调用除了console方式还有没有其他友好的方式\\
 & 写合约的语言有solidity,LLL,类python等,哪种是语言用得比较多,各个语言之间有什么差异\\
 & 开发智能合约有比较好用的IDE工具推荐吗,工作环境搭建上还不是很清楚\\
\hline
以太坊问题 & 以太坊每一笔交易基本上都要收费,这笔费用是按什么计价来收取的\\
 & 以太坊收费有没有其他方式来减少费用,比如打包交易等\\
 & 以太坊是底层是完全开源的吗\\
 & 以太坊上面跑的智能合约会越来越多,storage空间以后也会越来越大,这个问题以后会导致每一个节点同步很慢吗\\
 & 轻客户端有没有限制,可以任意地获取某些block之后的吗\\
\end{tabular}
\end{center}
\end{document}
